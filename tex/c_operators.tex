\chapter{C Operators}
Operators are the symbols we use to tell the compiler that we want to apply some operation on one or more numbers or variables.
The operations are generally arithmetic, logic, bitwise or comparison operations, but there are a few others as well.\\

\section{Arithmetic Operations}
In terms of arithmetic operations, the normal $+$, $-$, $*$ and $/$ apply. These operators are called binary operators because they are applied to two operands. In other words, there must be some number supplied on both sides of the $+$ symbol.

There are also two unary (only require 1 operand) arithmetic operators, the $++$ and $--$. These operators add 1 to or subtract 1 from the operand. If these operators are applied before a operand, it is called a pre-increment or pre-decrement. If they are applied after the operand it is called a post-increment or post-decrement. The difference is important to understand. 

\subsection{Post- vs Pre- increment/decrement}
Firstly, note that if the increment or decrement operators are the only thing being done in a statement then it makes no difference whether a post or pre operator is used. 
The difference only appears when there is a compound statement which requires the operand to be evaluated. The simplest case is that of an assignment. Clearly in order to assign the value of one variable (\texttt{foo}) to another (\texttt{bar}) the value of \texttt{foo} must be evaluated. 
Pre-increment means that the value is incremented and then the it is evaluated (ie: it is evaluated to the new value). 
Post-increment means that the value is evaluated and then incremented. 
This is demonstrated below.


\begin{lstlisting}[language=C]
// the following demonstrates pre-increment
// here, both foo and bar take on 0xAB
uint8_t foo = 0xAA;
uint8_t bar = ++foo;  // inc to 0xAB and then evaluate.

// the following demonstrates post-decrement.
// here, foo takes on 0x54 while bar takes on 0x55
uint8_t foo = 0x55;
uint8_t bar = foo--; // evaluate to 0x55 and then dec. 
\end{lstlisting}

\section{Logic Operations}
There are two classes of logic operands: bitwise logic and boolean logic. 
\subsection{Bitwise Logic}
Bitwise logic applies the logic operation to each individual bit of the operands. 
\begin{table}[h!]
\centering
\begin{tabu}{c c}
Operator & Meaning \\
\hline
\& & bitwise AND. Binary. \\
| & bitwise OR. Binary. \\
\textasciicircum & bitwise XOR. Binary\\
\textasciitilde & bitwise inverse (NOT). Unary\\
\end{tabu}
\caption{Commonly used bitwise operators.}
\end{table}

\subsection{Boolean Logic}
Boolean logic operators evaluate the operands as a boolean True or False and apply the operations on the operands as a whole.
These are most useful when you have multiple conditions which you want to turn into a compound condition. 
\begin{table}[h!]
\centering
\begin{tabu}{c c}
Operator & Meaning \\
\hline
\&\& & logical AND. Binary. \\
|| & logical OR. Binary. \\
! & logical inverse (NOT). Unary\\
\end{tabu}
\caption{Commonly used boolean operators.}
\end{table}

\section{Bit shift Operations}
Bit shift refers to moving all of the bits which represent a binary number up or down. 
The bits which fall off the end are simply thrown away. 
The empty spots are replaced with 0's. 
In order to shift bits up in C, the \verb|<<| operator is used. To shift bits down it is \verb|>>|. 
A bit shift up corresponds to a multiplication by a power of 2. For example, shifting bits up by 5 places is the same as multiplication by $2^5$.
This should be obvious if you write the numbers out in binary.
A bit shift down corresponds to a division by a power of 2, rounding the result DOWN. The result must be rounded down as extra bits (which can be thought of as the decimal component) are thrown away.
An example follows.

\begin{lstlisting}[language=C]
// set foo to equal bar times 128 (2^7)
foo = bar << 7;

// set foo to bar divided by 2 and rounded down.
foo = bar >> 1;
\end{lstlisting}

The reason why we may want to use bit shifts rather than multiplication or division operations is for efficiency. It is generally faster to perform a bit shift rather than a multiply or divide. 

The most frequent application of bit shifting for us is setting or clearing specific bits in registers. Consider the case where we want to set bit 27 of a register. The following shows a few ways of doing it.

\begin{lstlisting}[language=C]
someReg |= 134217728;
someReg |= 0b00001000000000000000000000000000;
someReg |= 0x80000000;
someReg |= (1 << 27);
\end{lstlisting}

The method with the bit shift is clearest in showing which bit is being set. 

\section{Typecast}
A typecast explicitly converts data of one type to another type. 
C is a weakly typed language, meaning that if you attempt to assign data of one type to a variable of another type it will do an implicit type conversion and perform the assignment.
This mean that typecasts are rarely (ever?) strictly necessary in the case of cross-type assignments.
Most often we do them for our own sake (improve code readability) or to prevent compile warnings.
If the compiler notices that you're assigning one data type to a different data type it may generate a warning to let you know that the result may not be what you expect.
This warning is resolved by an explicit typecast. 
The other key use of typecasting for us is to convert between different pointer types.
We know that the type of a pointer dictates how much memory will be accessed when the pointer is dereferenced, and also what amount will be added or subtracted from the pointer if it is incremented/decremented. 
Should we wish to change how a pointer behaves we can do so with a typecast. 
The following shows a few examples of where we may want typecasts.

\begin{lstlisting}[language=C]
// This would generate a warning as we're assigning a number to a pointer
uint16_t *foo_ptr = 0x2000 0040;
// No warning here was we explicitly cast the number to a pointer
// of the correct type.
uint16_t *foo_ptr = (uint16_t *)0x2000 0040;

// Again, warning as foo_prt and bar_ptr are different types
uint32_t *bar_ptr = foo_ptr;
// Warning resolved by telling compier we know what we're doing.
int32_t *bar_ptr = (int32_t *)foo_ptr;

// We only want to access the LSB of the data pointed to by foo_ptr.
// This is done by casing and then dereferencing.
*((uint8_t *)foo_ptr) = 0xAA;
\end{lstlisting}
